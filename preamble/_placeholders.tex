% Identificadores do trabalho
% Usados para preencher os elementos pré-textuais

\instituicao{Universidade Federal de Santa Catarina} % Opcional

\departamento{Departamento de Engenharia Elétrica}
\curso{Programa de Graduação em Engenharia Elétrica}
\documento{Monografia} % [o] para dissertação [a] para tese

\titulo{Estudo e Implementação de estação de carregamento de veículos elétricos}
\subtitulo{} % Opcional

\autor{Bruno Luiz da Silva}
\grau{Engenheiro Eletricista}

\local{Florianópolis} % Opcional (Florianópolis é o padrão)
\data{07}{Março}{2017}

\orientador{Prof. Dr. Eduardo Augusto {}Bezerra}
\coordenador{Prof. Renato Lucas Pacheco, Dr. Eng.}

\numerodemembrosnabanca{4} % Isso decide se haverá uma folha adicional
\orientadornabanca{sim} % Se faz parte da banca definir como sim
% \coorientadornabanca{nao} % Se faz parte da banca definir como sim
\bancaMembroA{Prof. Dr. Eduardo Augusto {}Bezerra\\Universidade Federal de Santa Catarina} % Nome do membro da Banca
\bancaMembroB{Dr. Cesare Quinteiro Pica\\Fundação CERTI} % Nome do membro da Banca
\bancaMembroC{Leonardo Kessler Slongo\\Universidade Federal de Santa Catarina} % Nome do presidente da banca
%\bancaMembroD{Quarto membro\\Universidade ...} % Nome do membro da Banca
%\bancaMembroE{Quinto membro\\Universidade ...} % Nome do membro da Banca
%\bancaMembroF{Sexto membro\\Universidade ...} % Nome do membro da Banca
%\bancaMembroG{Sétimo membro\\Universidade ...} % Nome do membro da Banca

% \dedicatoria{Dedicatório aqui}

%\agradecimento{}

\epigrafe{\textit{When Henry Ford made cheap, reliable cars people said, 'Nah, what's wrong with a horse?' That was a huge bet he made, and it worked}}{Elon Musk}
