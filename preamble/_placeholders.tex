% Identificadores do trabalho
% Usados para preencher os elementos pré-textuais

\instituicao{Universidade Federal de Santa Catarina} % Opcional

\departamento{Departamento de Engenharia Elétrica}
\curso{Programa de Graduação em Engenharia Elétrica}
\documento{Monografia} % [o] para dissertação [a] para tese

\titulo{Estudo e Implementação de estação de carregamento de veículos elétricos}
\subtitulo{} % Opcional

\autor{Bruno Luiz da Silva}
\grau{Engenheiro Eletricista}

\local{Florianópolis} % Opcional (Florianópolis é o padrão)
\data{15}{Dezembro}{2016}

\orientador{Eduardo Augusto {}Bezerra}
\coordenador{Prof. Renato Lucas Pacheco, Dr. Eng.}

\numerodemembrosnabanca{5} % Isso decide se haverá uma folha adicional
\orientadornabanca{sim} % Se faz parte da banca definir como sim
\coorientadornabanca{nao} % Se faz parte da banca definir como sim
\bancaMembroA{Prof. Dr. Jefferson Luiz Brum Marques\\Universidade Federal de Santa Catarina} % Nome do presidente da banca
\bancaMembroB{Prof. Dra. Daniela Ota Hisayasu Suzuki\\Universidade Federal de Santa Catarina} % Nome do membro da Banca
% \bancaMembroC{Eng. M.Sc. Sigmar de Lima\\Universidade Federal de Santa Catarina} % Nome do membro da Banca
%\bancaMembroD{Quarto membro\\Universidade ...} % Nome do membro da Banca
%\bancaMembroE{Quinto membro\\Universidade ...} % Nome do membro da Banca
%\bancaMembroF{Sexto membro\\Universidade ...} % Nome do membro da Banca
%\bancaMembroG{Sétimo membro\\Universidade ...} % Nome do membro da Banca

% \dedicatoria{Dedicatório aqui}

%\agradecimento{}

\epigrafe{Test}{Douglas Adams}
