\chapter{Revisão Bibliográfica}

  \section{Mercado de Veículos Elétricos}

  \section{Estações de carregamento - \ac{EVSE}}

    O carregamento de veículos elétricos é realizado por meio de estações de carregamento (\ac{EVSE}). O padrão IEC 62196 \cite{iec-62196} tenta padronizar alguns dos conectores já usados no mercado, assim como os modos de operações de uma estação de carregamento.


    \subsection{Modos de Carregamento Europeu}

      O modo de carregamento europeu segue o padrão IEC 62196 \cite{iec-62196}, que apresenta quatro modos de operação.

        \subsubsection{Modo 1 - Carregamento Lento}

        Utiliza plug de tomada residenciais, o qual não pode exceder 16 A de corrente e 250 V \ac{CA} monofásica ou 480 V AC de tensão trifásica. Normalmente ele toma de 6 à 8 horas para carregar completamente o veículo. Como é utilizada uma tomada comum, não é necessária uma \ac{EVSE} para fazer o controle desse carregamento. Porém, como esse padrão utiliza uma proteção diferencial e requer uma proteção aterrada na residência, muitos países não o adotam, visto que há boa parte das residências não possuem aterramento.

        \subsubsection{Modo 2 - Carregamento Lento}

        É necessária uma caixa de controle no plug ou no cabo, sendo que esta faz a comunicação com o veículo por meio de \ac{PWM}. Graças a essa comunicação, o carro pode informar o status de sua bateria e assim o carregamento pode ser adaptado de acordo com esse dado. Ele utiliza o mesmo plug do Modo 1 e possui as mesmas limitações elétricas de tensão, porém permite um máximo de até 32 A de corrente. Dentro dessa caixa de controle há também um circuito de proteção, o que permite o uso desse modo em locais não aterrados. Ele é utilizado em alguns locais públicos da Europa e é considerado uma solução de transição nos EUA.

        \subsubsection{Modo 3 - Carregamento Rápido CA}

        O cabo é conectado à uma \ac{EVSE}, sendo que ela precisa estar habilitada à comunicação PWM e possuir uma proteção elétrica. O carregamento pode ser ajustado de acordo com os dados recebidos pelo pino de controle da estação, possibilitando assim carregamentos lentos e rápidos. Usando uma tensão de 400 V trifásica com 63 A de corrente, é possível carregar certos carros em menos de uma hora. Esse modo está se tornando cada vez mais comum, porém requer um equipamento de eletrônica de potência adequado à tensão e corrente máxima desejadas. O desenvolvimento desse modo possui um grande investimento da empresa japonesa \textit{CHAdeMO}, sendo as vezes referido como padrão \textit{CHAdeMO}.

        \subsubsection{Modo 4 - Carregamento Rápido CC}

        Este modo ultra-rápido permite tensões de 400 V e corrente de 200 A, utilizando \ac{CC} para tal. Esse modo requer um inversor para converter a entrada da rede de CA para CC. Estações que permitem carregamento no modo 4 custam muito mais caro que estações modo 3, sem contar que o projeto precisa de uma atenção especial no quesito segurança.

    \subsection{Modos de Carregamento Americano}

      \begin{itemize}
        \item Nível 1: assim como o modo 1 europeu, utiliza um plug residencial (americano) para o carregamento, fornecendo até 120 V AC ao veículo.
        \item Nível 2: pode fornecer 240 ou 208 V AC e de 20 à 100 A. Em instalações residenciais, normalmente acaba limitado à 30 A, podendo oferecer até 7.2 kW de potência. Este é o modo mais comum de instalação em residências americanas.
      \end{itemize}

      Há ainda modos de corrente contínua, sendo que esses são idênticos ao Modo 4 Europeu.

    \subsection{Outros modos de carregamento}

      \subsubsection{Indução}

        O veículo pode ser carregado sem precisar estar conectado à uma estação, o que oferece maior segurança e comodidade para o motorista. Ele pode ser carregado de três maneiras

        \begin{itemize}
          \item Estática: veículo é carregado enquanto está estacionado
          \item Quasi-estática: o veículo é carregado enquanto há pessoas dentro, porém em locais específicos (parado no trânsito, por exemplo)
          \item Dinamicamente: enquanto o veículo está em movimento, como em uma rodovia
        \end{itemize}

        Ainda há muita pesquisa nesse modo, principalmente devido à sua mais baixa eficiência quando comparado aos modos cabeados.

      \subsubsection{Troca de Baterias}

        Há a possibilidade de carregamento por troca de baterias, onde o veículo para em uma estação e um sistema automatizado remove a bateria atual do carro e a substitui por uma carregada. Essa opção oferece segurança para o motorista e não sofre do problema de eficiência do carregamento por indução.

      \subsubsection{Tesla Supercharge}

        Nos modos cabeados, ainda há o modo proprietário da fabricante Tesla, o \textit{Tesla Supercharge}, que permite o carregamento de 50\% da bateria do Tesla S (85kWh) em menos de 20 minutos. Para possibilitar tal carregamento, os carregamentos são realizados em corrente contínua, assim como o Modo 4 Europeu.

    \subsection{Padrões de Conectores}

      Diversos tipos de conectores estão disponíveis hoje no mercado. Um dos padrões mais aceitos para o carregamento lento é o Tipo 2 - \textit{Mennekes}. Ele já foi submetido para se tornar padrão oficial desse tipo de carregamento na Europa \cite{mckinsey-report-ev}.

      Para carregamentos rápidos porém, existem três conectores que são bastante usados: o CHAdeMO, o Tipo 2 Combo e o Tesla Supercharger.

      \subsubsection{Tipo 2 - Mennekes}

        Proposto pela empresa Mennekes, permite carregamentos \ac{CA} monofásicos/trifásicos e \ac{CC}, além de ser retrocompatível com conectores Tipo 1, mesmo possuindo 7 pinos (contra 5 do tipo 1). Ele permite que a corrente flua de forma bi-direcional, o que permite que os \ac{EV} possam fornecer energia para a estação, previsto no modelo de \textit{Vehicle-to-Grid}). Ele é muito utilizado em carregamentos de modo 1 e modo 2, porém no modo 3 outros conectores são necessários.

      \subsubsection{Tipo 2 - Combo}

        Permite carregamentos rápidos em \ac{CC} e \ac{CA}. O conector possui uma parte eletrônica para garantir a segurança do usuário, onde o conector checa se foi conectado corretamente antes de a operação, além de previnir a desconecção durante o carregamento ou antes do pagamento ser efetivado.

      \subsubsection{CHAdeMO}

        Utilizado junto ao carregamento modo 3 e 4, foi o primeiro conector que possibilitou carregamentos rápidos e DC. Hoje, porém, outros padrões como o Tipo 2 Combo estão oferecendo um suporte satisfatório para carregamentos rápidos \cite{ieee-review-evse}. Visto a preocupação da CHAdeMO com segurança e o alto nível de tensão, esse conector possui 10 pinos.

      \subsubsection{Tesla Supercharger}

        Fornecido pela Tesla Motors, ele foca em carregamentos rápidos em \ac{CC} - modo 4 - e empresas como Nissan e BMW já negociam um acordo com a Tesla para a utilização do conector em seus veículos, permitindo que sua frota utilize a infra-estrutura de carregamentos Tesla Supercharge.

  \section{Protocolo OCPP}

    As estações de carregamento normalmente se comunicam com um servidor central, que pode gerenciar N estações. Para tal tarefa, é necessário um protocolo de comunicação. Embora ainda não exista um padrão oficial, o \ac{OCPP} é um padrão \textit{de facto} e já existem esforços para o tornar um padrão oficial junto a \ac{OASIS} \cite{ocpp-news-standardization}.

    Mantido e criado pela \ac{OCA}, o OCPP está presente em mais de 50 países. Na Europa, todas estações comercializadas precisam ser compatíveis com o OCCP e, na América, o interesse da indústria está aumentado \cite{ocpp-news-forbes}.

    O protocolo prevê um sistema central que recebe dados de N estações. Caso for necessário, o sistema central pode atuar sob \ac{EVSE} específicas com ações como reservar a estação, cancelar algum carregamento ou até desligá-la. Caso a estação perca conectividade, o protocolo prevê que ela deve funcionar de modo autônomo, somente registrando alguns dados para envio posterior (início e finalização de carregamentos) \cite{ocpp-spec-15}.

    As requisições da versão 1.5 podem ser via SOAP ou WebSocket, enquanto a versão 1.6 já adiciona suporte a JSON. A versão 2.0, em desenvolvimento, tem em mente a padronização do protocolo e retrocompatibilidade com versões anteriores.
