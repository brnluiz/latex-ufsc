\chapter{Conclusão}
\label{stateofart:conclusion}

  O objetivo de estudar o funcionamento de uma \textit{\ac{EVSE}} e como implementá-la foi alcançado com sucesso após os testes finais na estação protótipo.

  O \textit{software} desenvolvido nesse trabalho acadêmico é um dos elementos fundamentais para todo o projeto da estação. A estação em si é um produto pioneiro no Brasil, visto que essa é a primeira a ser desenvolvida em solo nacional (não há nenhum produto semelhante, desenvolvido por outras empresas nacionais), visando testes e análises de viabilidade para futuras oportunidades.

  % \item Desenvolver e aprender novas habilidades de engenharia de \textit{software}
  A complexidade do \textit{software}, assim como todas tecnologias utilizadas, permitiram que os objetivos relacionados à \textit{software} fossem alcançados também. Embora houveram problemas ao transferir a aplicação da bancada de testes para a estação, estes permitiram o aprendizado de métodos para resolução de problemas.

  % \item Aprimorar técnicas de desenvolvimento de \textit{software} embarcado
  Durante o projeto, alguns dos desafios encontrados foram relacionados a camadas de software entre o sistema operacional e o hardware. Embora tenham desacelerado um pouco o projeto, esses desafios permitiram o desenvolvimento de um conhecimento de Linux embarcado, que possui algumas peculiaridades quando comparado a versão \textit{desktop}, como \textit{Device Trees}, gerenciador de boot diferenciado (\textit{GRUB} x \textit{uBoot}) e acesso facilitado a recursos da placa utilizada (periféricos, entradas/saídas e serial).

  Uma das sugestões de modificação do projeto é a utilização de medidores integrados à BeagleBone por meio das entradas analógicas ou digitais (caso a saída do medidor for codificada). Isso facilitaria o desenvolvimento futuro de uma placa mais otimizada, com todos dispositivos integrados à mesma (micro-controlador, comunicação serial, acionamentos e entradas).
  
  % \item Conhecer mais sobre as tecnologias envolvidas em sistemas de veículos elétricos
  % \item Implementar um \textit{software} para controle de uma \textit{\ac{EVSE}}
  % \item Explorar e implementar protocolos de comunicação de \textit{\ac{EVSE}}
  O mercado de veículos elétricos se mostra uma área promissora para qualquer engenheiro eletricista, visto que apresenta desafios e oportunidades em diferentes competências. Embora o foco desse projeto tenha sido a implementação do protocolo e do \textit{software} de controle, foi possível não só aprender sobre isso, mas também sobre outros aspectos desse mercado, visto que no ambiente de trabalho houveram diversas conversas sobre assuntos relacionados (regulamentação, equipamentos, novas tecnologias, tendências e entre outros).

  \section{Reconhecimento}
  
    Os trabalhos descritos neste documento foram realizados no âmbito do projeto "Sistema de Recarga Rápida com Armazenamento Híbrido-Estacionário de Energia para Abastecimento de Veículos Elétricos no Conceito de Redes Inteligentes" (P\&D ANEEL 5697-0414/2014), financiado pelo programa de P\&D da ANEEL, viabilizado pela CELESC Distribuição S.A. e executado pela Fundação CERTI.
  