\chapter{Introdução}
\label{stateofart:intro}

  Desde o início dessa década, as empresas automobilísticas perceberam o grande potencial que o mercado de veículos elétricos possui. Antes, esse mercado era limitado por questões tecnológicas, como durabilidade e carga de baterias,  sistemas embarcados e eletrônica de potência. Porém, hoje essas limitações vem sendo superadas, o que re-aqueceu o setor. O carregamento desses veículos é dado por \textit{\ac{EVSE}}.

  Uma \textit{\ac{EVSE}} traz não só desafios na área de eletrônica de potência, mas também na área de sistemas embarcados e \textit{software} em geral, visto que a estação pode ser feita de múltiplos dispositivos que precisam comunicar-se entre si e com servidores na nuvem. Esse projeto explora a área do \textit{software} de uma \textit{\ac{EVSE}}.

  \section{Motivação}
  \label{stateofart:intro:motivation}

    Existem diversos aspectos que podem ser tratados nesse tipo de projeto. Há, dentro da estação, dispositivos que permitem medir e atuar sobre os carregamentos. Todos precisam ser controlados por um \textit{software}, o que requer do desenvolvedor conhecimentos de comunicação e projeto de \textit{software}. Além disso, a maioria dos sistemas contemporâneos são conectados a servidores externos, e no caso das \textit{\ac{EVSE}} não é diferente.

    Este trabalho acadêmico explora a implementação do \textit{software} que integra os periféricos da \textit{\ac{EVSE}} e realiza a comunicação entre a estação de carregamento e um sistema central na nuvem. Tais tarefas permitem um aprendizado mais profundo sobre engenharia de \textit{software}, protocolos de comunicação e integração de múltiplos dispositivos.

  \section{Objetivos}
  \label{stateofart:intro:objectives}

    Este Trabalho de Conclusão de Curso tem como objetivo apresentar e implementar um protótipo de estação de carregamento elétrica, que permitirá ao usuário iniciar e cancelar carregamentos, assim como visualizar dados do conector: status (carregando, disponível, veículo conectado) e, caso estiver em um carregamento, tempo total e consumo de energia. Os carregamentos fornecidos pela estação são \ac{AC} e esses podem ser realizados somente em veículos que estejam conforme a norma IEC 62196 \cite{iec-62196}. A estação não fornece carregamentos \ac{CC}, o que permitiria carregamentos mais rápidos porém subiria os custos do protótipo.

    A comunicação com o servidor também é implementada, de modo que este possa receber leituras dos sensores e atualizações dos carregamentos, assim como possibilitar o gerenciamento remoto da estação. Essa comunicação permite a inicialização e finalização de carregamentos de modo remoto, o armazenamento de medições dos carregamentos (corrente, tensão, potência e energia), a verificação do estado dos conectores, a configuração de variáveis internas da estação entre outros. No futuro, tais medições poderão ser utilizadas para efetivar cobranças ao usuário, verificar e atuar remotamente em casos de falhas e permitir que usuários iniciem seus carregamentos por meio de aplicativos de celular.

    Para isso são utilizadas tanto tecnologias já consolidadas, assim como outras mais recentes. O projeto é implementado em uma placa de desenvolvimento que possibilita acesso remoto, rodando o sistema operacional \textit{Linux}.

    O objetivo do acadêmico neste projeto não é implementar a parte elétrica da estação, mas sim seu \textit{software}. A parte elétrica foi desenvolvida por outra equipe.

  \section{Objetivos Específicos}
  \label{stateofart:intro:specificobjectives}

    \begin{itemize}
      \item Aprimorar técnicas de desenvolvimento de \textit{software} embarcado
      \item Conhecer mais sobre as tecnologias envolvidas em sistemas de veículos elétricos
      \item Explorar e implementar protocolos de comunicação de \textit{\ac{EVSE}}
      \item Desenvolver e aprender novas habilidades de engenharia de \textit{software}
      \item Implementar um \textit{software} para controle de uma \textit{\ac{EVSE}}
    \end{itemize}
