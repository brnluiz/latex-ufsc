\chapter{Introdução}
\label{stateofart:intro}

  Desde o início dessa década, as empresas automobilísticas perceberam o grande potencial que o mercado de veículos elétricos possui. Antes, esse mercado era limitado por questões tecnológicas relacionadas a bateria e sistemas embarcados, porém hoje essas limitações vem sendo superadas, o que re-aqueceu o setor. O carregamento desses veículos é dado por \textit{\ac{EVSE}}.

  Uma \textit{\ac{EVSE}} traz não só desafios na área de eletrônica de potência, mas também na área de sistemas embarcados e \textit{software} em geral, visto que a estação é feita de múltiplos dispositivos que precisam comunicar-se entre si e com servidores na nuvem. Esse projeto explora a área do \textit{software} de uma \textit{\ac{EVSE}}.

  \section{Motivação}
  \label{stateofart:intro:motivation}

    Existem diversos aspectos que podem ser tratados nesse tipo de projeto. Há, dentro da estação, dispositivos que permitem medir e atuar sobre os carregamentos. Todos precisam ser controlados por um \textit{software}, o que requer do desenvolvedor conhecimentos de comunicação e projeto de \textit{software}. Além disso, a maioria dos sistemas contemporâneos são conectados a servidores externos, e no caso das \textit{\ac{EVSE}} não é diferente.

    Este trabalho acadêmico explora a implementação do \textit{software} que integra os periféricos da \textit{\ac{EVSE}} e realiza a comunicação entre a estação de carregamento e um sistema central na nuvem. Tais tarefas permitem um aprendizado mais profundo sobre engenharia de \textit{software}, protocolos de comunicação e integração de múltiplos dispositivos.

  \section{Objetivos}
  \label{stateofart:intro:objectives}

    Este Trabalho de Conclusão de Curso tem como objetivo apresentar e implementar um protótipo de estação de carregamento elétrica, que permitirá ao usuário iniciar e cancelar carregamentos, assim como visualizar os dados correspondentes.

    A comunicação com o servidor também será implementada, de modo que este possa receber leituras dos sensores e atualizações dos carregamentos, assim como possibilitar o gerenciamento remoto da estação.

    Para isso serão utilizadas tanto tecnologias já consolidadas, assim como outras mais recentes. O projeto será implementado em uma placa de desenvolvimento que permita acesso remoto (via \textit{Ethernet}), rodando o sistema operacional \textit{Linux}.

  \section{Objetivos Específicos}
  \label{stateofart:intro:specificobjectives}

    \begin{itemize}
      \item Aprimorar técnicas de desenvolvimento de \textit{software} embarcado
      \item Conhecer mais sobre as tecnologias envolvidas em sistemas de veículos elétricos
      \item Explorar e implementar protocolos de comunicação de \textit{\ac{EVSE}}
      \item Desenvolver e aprender novas habilidades de engenharia de \textit{software}
      \item Implementar um \textit{software} para controle de uma \textit{\ac{EVSE}}
    \end{itemize}
