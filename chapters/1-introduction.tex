\chapter{Introdução}

Desde o início dessa década, empresas automobilísticas veem o mercado de veículos elétricos com outros olhos. Antes, limitado por questões como bateria e sistemas embarcados, estes não o são mais, o que re-aqueceu o setor.

Uma estação de carregamento elétrico (EVSE) traz não só desafios na área de eletrônica de potência, mas também na área de sistemas embarcados e \textit{software} no geral, visto que a estação é feita de múltiplos dispositivos que precisam comunicar-se entre si e com servidores na nuvem. Esse projeto explorará a área do \textit{software} de uma EVSE.

\section{Motivação}

Existem diversos aspectos que podem ser tratados nesse tipo de projeto. Há, dentro da estação, dispositivos que permitem medir e atuas sob os carregamentos. Todos precisam ser controlados por um \textit{software}, o que requer do desenvolvedor conhecimentos de comunicação e projeto de \textit{software}. Além disso, hoje qualquer sistema moderno é conectado a servidores externos, e no caso das EVSEs não é diferente.

Este trabalho acadêmico explora a implementação do \textit{software} que integra os periféricos da EVSE e realiza a comunicação entre o posto e um sistema central na nuvem. Tais tarefas permitem um aprendizado mais profundo sobre engenharia de \textit{software}, protocolos de comunicação e integração de múltiplos dispositivos.

\section{Objetivos}

Este Trabalho de Conclusão de Curso tem como objetivo apresentar e implementar, um protótipo de estação de carregamento elétrica, que permitirá ao usuário iniciar e cancelar carregamentos, assim como visualizar os dados do carregamento atual.

A comunicação com o servidor também será implementada, de modo que este possa receber leituras dos sensores e atualizações dos carregamentos, assim como possibilitar o gerenciamento remoto da estação.

Para isso serão utilizadas tanto tecnologias já consolidadas na indústria, assim como tecnologias mais recentes. O projeto todo será implementado em um sistema embarcado de modo que esse possibilite o acesso remoto à estação.

\section{Objetivos Específicos}

\begin{itemize}
  \item Aprimorar técnicas de desenvolvimento de \textit{softwares} embarcados
  \item Conhecer mais sobre as tecnologias envolvidas em sistemas de veículos elétricos
  \item Conhecer sobre mais sobre os atuais protocolos de comunicação entre EVSEs e nuvem
  \item Desenvolver e aprender novas habilidades de engenharia de \textit{software}
  \item Implementar um \textit{software} para controle de uma estação de carregamento de veículos elétricos
\end{itemize}
